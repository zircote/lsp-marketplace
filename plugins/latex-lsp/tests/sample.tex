% Hook test
% Sample LaTeX document for LSP plugin validation
%
% This file contains various LaTeX constructs to test:
% - LSP operations (hover, go to definition, completion)
% - Hook validation (chktex linting)
% - Document structure and organization

\documentclass[12pt,a4paper]{article}

% Packages
\usepackage[utf8]{inputenc}
\usepackage{amsmath}
\usepackage{amssymb}
\usepackage{graphicx}
\usepackage{hyperref}
\usepackage{listings}
\usepackage{xcolor}

% Document metadata
\title{Sample LaTeX Document}
\author{Claude Code}
\date{\today}

% Custom commands
\newcommand{\R}{\mathbb{R}}
\newcommand{\N}{\mathbb{N}}
\newcommand{\vect}[1]{\mathbf{#1}}

\begin{document}

\maketitle

\begin{abstract}
This is a sample LaTeX document designed to test LSP integration, including document structure, mathematical typesetting, cross-references, and citations.
\end{abstract}

\tableofcontents

\section{Introduction}
\label{sec:intro}

This document demonstrates various \LaTeX{} features for testing the Texlab LSP integration. It includes sections, mathematical equations, figures, tables, and cross-references.

% TODO: Add more introductory content
% FIXME: Improve abstract

\section{Mathematical Content}
\label{sec:math}

\subsection{Inline and Display Mathematics}

Consider the function $f: \R \to \R$ defined by $f(x) = x^2 + 2x + 1$. This function can be written in factored form as:

\begin{equation}
f(x) = (x + 1)^2
\label{eq:quadratic}
\end{equation}

The derivative of $f$ is given by:
\begin{equation}
f'(x) = 2x + 2 = 2(x + 1)
\label{eq:derivative}
\end{equation}

\subsection{Matrix Notation}

A general $n \times n$ matrix can be represented as:

\begin{equation}
A = \begin{pmatrix}
a_{11} & a_{12} & \cdots & a_{1n} \\
a_{21} & a_{22} & \cdots & a_{2n} \\
\vdots & \vdots & \ddots & \vdots \\
a_{n1} & a_{n2} & \cdots & a_{nn}
\end{pmatrix}
\label{eq:matrix}
\end{equation}

\subsection{Theorem Environment}

\begin{theorem}[Pythagorean Theorem]
For a right triangle with legs of length $a$ and $b$, and hypotenuse of length $c$:
\begin{equation}
a^2 + b^2 = c^2
\label{eq:pythagoras}
\end{equation}
\end{theorem}

\section{Lists and Enumerations}

\subsection{Itemized List}

Key features of \LaTeX:
\begin{itemize}
    \item Professional typesetting
    \item Excellent mathematical notation
    \item Automatic numbering and cross-referencing
    \item Bibliography management
    \item Separation of content and presentation
\end{itemize}

\subsection{Numbered List}

Steps in the document workflow:
\begin{enumerate}
    \item Write content in \texttt{.tex} file
    \item Compile with \texttt{pdflatex}
    \item Review output PDF
    \item Iterate as needed
\end{enumerate}

\section{Cross-References}

As shown in Equation~\ref{eq:quadratic}, the quadratic function has a simple form. Its derivative (Equation~\ref{eq:derivative}) is linear. The general matrix form is presented in Equation~\ref{eq:matrix}.

Section~\ref{sec:intro} provides an introduction, while Section~\ref{sec:math} covers mathematical content.

\section{Tables and Figures}

\subsection{Tables}

Table~\ref{tab:sample} shows sample data.

\begin{table}[h]
\centering
\begin{tabular}{|l|c|r|}
\hline
\textbf{Item} & \textbf{Value} & \textbf{Unit} \\
\hline
Length & 10.5 & cm \\
Width & 5.2 & cm \\
Height & 3.8 & cm \\
\hline
\end{tabular}
\caption{Sample measurement data}
\label{tab:sample}
\end{table}

\subsection{Figures}

% Note: This would normally include a real figure
% \begin{figure}[h]
% \centering
% \includegraphics[width=0.8\textwidth]{example-image}
% \caption{Sample figure}
% \label{fig:sample}
% \end{figure}

\section{Code Listings}

Example Python code:

\begin{lstlisting}[language=Python, caption={Hello World in Python}]
def greet(name: str) -> str:
    """Return a greeting message."""
    return f"Hello, {name}!"

if __name__ == "__main__":
    print(greet("World"))
\end{lstlisting}

\section{Conclusion}

This sample document demonstrates the key features of \LaTeX{} that should be supported by the LSP integration, including structure navigation, cross-references, mathematical notation, and syntax highlighting.

\end{document}
